\newpage


\chapter{Problémová oblasť}
\label{analyza_problemova_oblast}

Problémovú oblasť v tejto práci predstavuje výber produktov a služieb v internetovom predaji, ktoré majú maximálny potenciál zaujať zákazníkov pri práci s predajným portálom. Na takéto vzťahy medzi produktami a zákazníkmi alebo produktami navzájom môže mať vplyv množstvo faktorov ako napríklad ročné obdobie, vek a pohlavie nakupujúceho, špeciálne zľavy.
Táto problematika zahŕňa podproblémy, ktorým sa venujeme ako analýza nákupného košíka(Market basket analysis - ďalej ako MBA) alebo cross-selling.
 \newline

\section{Odporúčacie systémy}
\label{odporucacie_systemy}

Odporúčacie systémy(ang. \textit{recommenders}) predstavujú skupinu softvérových nástrojov pre odporúčanie položiek používateľom. Ako najčastejšie aplikácie v súčasnosti vystupuje odporúčanie v online nakupovaní, hudbe alebo zdrojoch informácií ~\cite{ricci2011introduction}. Odporúčacie systémy sú obľúbenou témou výskumu kvôli svojmu potenciálu pri aplikácii v biznis sfére.

Hlavnou úlohou odporúčacieho systému je poskytnúť odporúčania - zoradený zoznam položiek, kde nás zaujímajú položky s najvyšším hodnotením. Hodnotenie(angl. \textit{rating}) predstavuje relevantnosť alebo zaujímavosť, ktorú pre používateľa daná položka má. Ako jednoduchý príklad hodnotenia môžeme uviesť skóre alebo počet hviezdičiek, ktoré používateľ priradí filmu, reštaurácii alebo hotelu. Odporúčacie systémy sa snažia odhaliť charakteristické črty cieľového používateľa a vygenerovať zoznam, ktorý zodpovedá jeho potrebám alebo záujmom ~\cite{adomavicius2005toward}.

\subsection{Odporúčacie systémy založené na obsahu}
\label{content_based_recommenders}

Táto podmnožina odporúčacích systémov je založená na myšlienke, že používateľovi sa budú páčiť položky, ktoré zdieľajú charakteristické črty s položkami, o ktoré prejavil záujem v minulosti. Úlohou systému je teda evidencia charakteristických čŕt položiek a generovanie matíc vzdialeností medzi jednotlivými položkami. Výber odporúčaných položiek je následne realizovaný ako zoradený zoznam položiek podľa najkratšej vzdialenosti od množiny položiek, ktoré používateľ ohodnotil pozitívne a zároveň najväčšej vzdialenosti od negatívne hodnotených položiek~\cite{adomavicius2005toward}.


\subsection{Kolaboratívne odporúčacie systémy}
\label{collaboration_based_recommenders}

Narozdiel od predošlého prístupu, kolaboratívne systémy sa spoliehajú na nepriame spojitosti medzi položkami. Analýzou predošlých preferencií sa vytvorí profil používateľa, ktorý je následne porovnávaný s profilmi ostatných používateľov. Vytvorený zoznam odporúčaní predstavuje najlepšie ohodnotené položky od používateľov s najväčšou zhodou používateľského profilu~\cite{adomavicius2005toward}.

\subsection{Hybridné systémy}
\label{hybrid_recommenders}

Kombinácie prístupov sa snažia prekonať nedostatky oboch prístupov vzájomným dopĺňaním sa. 

Existuje niekoľko prístupov, ktorými je možné kombinovať prístupy ~\cite{burke2002hybrid}:

\begin{my_itemize}
	\item{Váhovanie} - kombinácia výstupov oboch systémov do jedného odporúčania
	\item{Prepínanie} - vyberanie výstupu podľa situácie
	\item{Miešanie} - zobrazenie viacerých výstupov separátne
	\item{Kombinovanie} - spájanie podčastí do jediného systému
	\item{Kaskáda} - systém B schvaľuje položky vybrané iným systémom
	\item{Augmentácia} - systém B odporúča výber z výstupu systému A
	\item{Meta-úroveň} - model systému A je vstupom pre systém B
\end{my_itemize}

\subsection{Problémy odporúčania}
\label{recommender_problems}

Slabinou odporúčacích systémov sú nedostatočné historické dáta. Problém predstavujú hlavne pre nových používateľov, ktorí si nevybrali dostatok položiek aby im bol vygenerovaný používateľský profil. V takomto prípade nie je možné robiť porovnanie s ostatnými používateľmi ani generovať na základe už vybraných položiek. Je to problémom kolaboratívneho aj obsahového odporúčania. Podobný problém predstavuje ohodnotená vzorka položiek. Je možné generovať odporúčania podobných položiek, ale bez ohodnotenia svojej skúsenosti predchádzajúcim používateľom nie je nijaká záruka, že odporúčanie bude vhodné~\cite{adomavicius2005toward}.

Nedostatky prístupov vznikajú z neschopnosti nájsť asociácie medzi používateľmi a produktami:

\begin{my_itemize}
	\item{Problém nového používateľa} - neschopnosť priradiť používateľa k iným používateľom pre nedostatok porovnateľných preferencií
	\item{Problém nového produktu} - neschopnosť odporučiť produkt, ktorý sa zatiaľ nenachádza u iných používateľov
	\item{Problém riedkosti} - kolaboratívne systémy vyžadujú kritickú masu používateľov, pri ktorej začínajú byť efektívne. Počet ohodnotení produktov je navyše často nepomerne menší k počtu odporúčaní, ktoré je nutné generovať.
\end{my_itemize} 

\chapter{Neurónové siete}
\label{analyza_neuronove_siete}

Koncept neurónových sietí vznikol v 40. rokoch minulého storočia inšpiráciou biologickými neurónovými sieťami v mozgu~\cite{mcculloch1943logical}.
Cieľom bolo prekonať bariéru medzi tým, čo je pre ľudský mozog ľahko riešitelné ale ťažko formálne definovatelné matematickými pravidlami. Tieto problémy, ktoré riešime intuitívne, pri pokuse o formálnu špecifikáciu ukazujú, aké množstvo znalostí používame v každodennom živote. Ako vhodný príklad slúži vizuálne rozoznávanie objektov, ktoré je pre osobu samozrejmé, no až v posledných rokoch zaznamenávame prvé úspechy v tejto problematike pri použití NN~\cite{Goodfellow-et-al-2016-Book}.

\section{Štruktúra}
\label{analyza_struktura_nn}

Podobne ako v mozgu, základ neurónovej siete tvoria neuróny a prepojenia medzi nimi. Neuróny sú organizované vo vrstvách, ktoré sa delia na 3 základné typy. 
%\newline
\noindent

\textbf{Vstupná vrstva} - reprezentuje dáta, ktoré podsúvame sieti pre interpretáciu. Dáta musia byť pred posunutím vstupnej vrstve často predspracované, aby bola sieť schopná interpretovať ich. Počet neurónov na vstupnej vrstve je ovplyvnený množstvom dát, ktoré máme na vstupe. V sieti existuje iba jediná vstupná vrstva.
%\newline
\noindent

\textbf{Výstupná vrstva} - interpretácia dát sieťou. Výstupnú vrstvu je možné nazvať ,,výsledok"  siete.
\noindent

\textbf{Skrytá vrstva} - nachádzajú sa medzi vstupnou a výstupnou vrstvou. Ich počet určuje hĺbku siete. NN nemusí mať ani jednu skrytú vrstvu, no takáto sieť dokáže modelovať iba lineárnu závislosť. Všeobecne platí, že čím viac skrytých vrstiev má sieť, tým zložitejšie vzťahy dokáže simulovať. Zvyšujú sa však aj nároky na učenie a výpočtové nároky. Jediná skrytá vrstva vytvára pozoruhodný rozdiel v aplikovatelnosti modelu, keďže prekonáva hranicu lineárnej závislosti funkcie, ktorú model pokrýva. Pri vysokej zložitosti modelu je možné naraziť na problém preučenia, ktorý bráni sieti korektne generalizovať. Neexistuje nijaký spoľahlivá metóda pre správny počet alebo veľkosť skrytých vrstiev. Empiricky sa vyvinulo niekoľko odhadov, ale v praxi je nutné overovať správnosť modelu praktickou evaluáciou. Odhadové pravidlá najčastejšie padajú na neschopnosti integrovať vo svojom rozhodnutí komplexitu úlohy a redundanciu v tréningových dátach ~\cite{Goodfellow-et-al-2016-Book}.
%\newline
\noindent

\textbf{Prepojenia} - Váhované prepojenia medzi neurónmi fungujú ako pamäť neurónovej siete. V jednoduchom modeli neurónovej siete sú prepojenia iba medzi neurónmi navzájom susediacich vrstiev. Prepojenie existuje medzi každým neurónom \textit{n}-tej do \textit{n+1} vrstvy. Neuróny jednej vrstvy pritom medzi sebou nie sú prepojené. Signál sa šíri týmito prepojeniami od vstupnej vrstvy smerom k výstupnej vrstve v jednom smere, ako je to ilustrované na obr.~\ref{fig:fnn}. Takéto siete sa volajú \textit{dopredné}. Hlavný účel prepojenia je niesť váhu. Váha prepojenia určuje, aký významný je vzťah medzi dvomi danými neurónmi, ktoré spája. Korektná váha daného prepojenia je na začiatku neznáma, jej korektné nastavenie je výsledkom procesu učenia~\cite{Goodfellow-et-al-2016-Book}.
%\newline
\noindent

\textbf{Neurón} - predstavuje základnú stavebnú jednotku neurónovej siete. Skladá sa z \textit{aktivačnej funkcie} a \textit{prahovej hodnoty}. Prahová hodnota neurónu $\vartheta _{i}^{k+1}$  je odpočítaná od sumy vstupných váhovaných hodnôt $ w_{ij}^{k}\, .\, o_{j}^{k}$. 
\newline
Na výsledok $o_{i}^{k+1}$ sa následne aplikuje aktivačná funkcia $f$ podľa obr.~\ref{fig:neuronoutput}. Takýto výstup je následne prepojeniami posielaný do ďaľších neurónov. Špeciálny prípad je neurón vstupnej a výstupnej vrstvy. Na vstupe totiž neurón hodnotu iba posiela ďalej a na výstupe po spracovaní nie je zasielaná nikam - predstavuje výsledok siete.
\newline



%\[o_{i}^{k+1}= f\left ( \sum_{j=1}^{N} w_{ij}^{k}\, .\, o_{j}^{k}\, -\, \vartheta _{i}^{k+1} \right )\]


\begin{figure}[H]
\begin{center}\includegraphics[scale=0.64]{neuronoutput}\end{center}
\caption[neuronoutput]{Výstupná hodnota neurónu~\cite{kvasnivcka1997uvod}}\label{fig:neuronoutput}
\end{figure}

%\newline
\noindent



\begin{figure}[H]
\begin{center}\includegraphics[scale=0.64]{fnn}\end{center}
\caption[fnn]{Štruktúra doprednej neurónovej siete (FNN)~\cite{jaeger2002tutorial}}\label{fig:fnn}
\end{figure}

\section{Učenie neurónovej siete}
\label{analyza_ucenie_nn}

Učenie predstavuje kľúčovú aktivitu pre schopnosť siete produkovať požadované výsledky. Spočíva vo vystavovaní neurónovej siete tréningovým dátam, ktoré sa sieť snaží interpretovať.

\textbf{Učenie s učiteľom} je metóda, pri ktorej je dostupná sada tréningových dát ,,označená", t.j. je dostupná hľadaná informácia/správne riešenie, ktoré má neurónová sieť generovať. Pri interpretovaní výsledku je možné okamžite určiť, aká chyba nastala a následne ju propagovať do siete. Na toto sa využíva tzv. \textit{spätná propagácia}(backpropagation), ktorá upravuje váhy siete v rozsahu chyby, ktorá nastala - rozdiel medzi správnym výsledkom pre daný vstup a samotným výsledkom siete.
\noindent

\textbf{Učenie bez učiteľa} predstavuje alternatívnu metódu, pri ktorej tréningové dáta nemajú dostupné výsledky. Neurónová sieť sa sama učí rozhodnúť, čo je pre ňu relevantné. Učenie bez učiteľa predstavuje možnosť ako získať takmer neobmedzené množstvá tréningových dát tam, kde učenie s učiteľom vyžaduje manuálne a kvôli časovej náročnosti nedostupné označovanie.
\noindent

\section{Optimalizácia}
\label{optimalization}

Snahou optimalizácie je hľadať ideálne riešenie v často obrovskom prehľadávanom priestore. V prípade neurónových sietí je to hľadanie optimálneho nastavenia siete, ktorá následne dokáže meniť vstup na predpokladaný výstup. Asi najčastejšou úlohou optimalizačných techník v neurónových sieťach je minimalizovať stratu(angl. \textit{loss}). Tá naznačuje, ako ďaleko sa naša sieť nachádza od konvergencie k ideálnemu riešeniu. Je definovaná ako súčet chýb, ktoré boli dosiahnuté v tréningových alebo validačných množinách. Pri učení sa snažíme optimalizovať stratu, ale nie nulovať ju. V bežných prípadoch pri zašumených dátach totiž nulová chyba znamená preučenie - kopírovanie datasetu namiesto odhaľovania vzorov v ňom.

\subsection{Gradient descent}
\label{stochastic_gradient_descent}

Gradient descent predstavuje stratégiu spätnej propagácie v ktorej je vypočítaná derivácia chyby a tá je následne prenesená do parametrov siete ich upravovaním pre minimalizáciu chyby v nasledujúcej iterácii. 

Obľúbená varianta je \textbf{stochastický gradient descent}, ktorý je síce menej presný, ale prenáša chybu po každej iterácii, narozdiel od originálneho stochastického gradientu, ktorý upravuje parametre až po celej sade. Ajkeď teda SGD nedosiahne presnosť GD, dostane sa do blízkosti riešenia oveľa rýchlejšie a bude tam oscilovať. Táto vlastnosť ho uprednostňuje pri väčších datasetoch.

Pri používaní gradient descent algoritmu je problematické správne odhadnúť hyperparametre. Bežný postup je preskúmať široké okolie a sledovať ako vplýva zmena hyperparametrov na schopnosť minimalizácie chyby~\cite{zhang2004solving}.


\section{Hyperparametre}
\label{analyza_hyperparametre}

Nastavenia, pomocou ktorých kontrolujeme správanie neurónových sietí sa nazývajú \textit{hyperparametre}. Tieto hodnoty nie sú získané učením siete pokiaľ nemodelujeme vnorený systém za týmto účelom. Príkladom hyperparametra je počet skrytých vrstiev NN. 

 Medzi ďaľšie významné hyperparametre patria:
 
\subsubsection*{Šírka a počet skrytých vrstiev} Pamäť neuurónovej siete je zaznamenávaná neurónmi a váhovanými prepojeniami medzi nimi. Váhy upravované počas učenia nesú informácie, ktoré definujú náš model. Pri nízkom počte nebude model schopný naučiť sa funkciu definovanú problémom. Pri vysokom počte je možné, že sieť v sebe uloží menší tréningový dataset, nazývané tiež ako problém \textit{preučenia}(overfitting). Pri preučení sieť nezíska schopnosť generalizácie problému kvôli sledovaniu tréningového datasetu. Je zjavné, že zvolenie správnych hyperparametrov má pre výsledky metódy kľúčovú úlohu~\cite{Goodfellow-et-al-2016-Book}.

\subsubsection*{Rýchlosť učenia} 
Rýchlosť učenia(learning rate) rozhoduje o tom, v ako rozsahu bude upravovaná sieť pri spätnej propagácií. Vystupuje ako kvantifikátor aplikovanej zmeny váh.

\subsubsection*{Miera výpadku neurónov}(dropout rate) - určuje, aká časť neurónov nebude pri tréningu použitá v každej etape. Náhodne vybrané neuróny následne nepropagujú signál, čím nútia neurónovú sieť nespoliehať sa na konkrétne nastavenie, čím sa redukuje problém preučenia a zlepšuje schopnosť generalizovať vzory v dátach~\cite{srivastava2014dropout}.
\subsubsection*{Chybová funkcia}(loss function) - určuje mieru chyby, ktorá nastáva v priebehu učenia. Čím väčšia chyba, tým väčšie zmeny sú propagované pri učení siete. Miera chyby by sa v priebehu učenia mala graduálne zmenšovať v priebehu konvergencie modelu k správnemu riešeniu. Správna chybová funkcia je teda kritickou súčasťou modelu~\cite{christoffersen2004importance}.
\subsubsection*{Momentum} - predstavuje riešenie pre problém lokálneho minima aj pre osciláciu pri stochastickom gradient descente. Vysoké momentum je schopné prekonať lokálne minimum a pri oscilácii okolo globálneho riešenia sa spomaľuje~\cite{attoh1999analysis}.

\subsubsection*{Aktivačná funkcia}
\label{activation_function}

Aktivácia je matematická operácia aplikovaná na výstupy z predchádzajúcej vrstvy. Mení vstupné hodnoty na výstupný signál. 

\textbf{Sigmoidná funkcia} - Produkuje signál v kladnom rozsahu $<0,1>$. Najefektívnejšia je pre dáta ktoré sú na vstupe v rovnakom rozsahu~\cite{sibi2013analysis}.

\textbf{ReLU} - Usmernená lineárna jednotka, predstavuje najjednoduchšiu derivovateľnú nelineárnu funkciu. Nesaturuje výstup a vďaka tomu dosahuje dobré výsledky pre hlboké neurónové siete - nevzniká efekt miznúceho gradientu.
	
\textbf{Softmax} - Funkcia najčastejšie využívaná pri klasifikačných úlohách na výstupe. Škáluje výsledné neuróny tak, aby spoločný výstupný signál dosahoval hodnotu 1. ~\cite{toth2013phone}


\section{Rekurentné neurónové siete}
\label{analyza_pokrocile_modely_nn}

Do popredia výskumu sa v súčasnosti dostávajú pokročilé modely, ktoré už nie sú obmedzené na jednoduchý dopredný prístup. Vďaka rapídnemu zvyšovaniu výkonu grafických kariet sa čoraz častejšie aplikujú \textit{rekurentné modely neurónových sietí}(RNN)~\cite{jaeger2002tutorial}. Špecializáciou rekurentných sietí je práca so sekvenčnými dátami. Tieto siete predstavujú generalizáciu dopredných modelov ich rozšírením o cyklické prepojenia~\cite{Goodfellow-et-al-2016-Book}.
Takýmto spôsobom je možné využiť súčasnú hodnotu premennej na ovplyvnenie vlastnej hodnoty v budúcnosti. Cyklický charakter rekurentného modelu je zobrazený na obr.~\ref{fig:rnn}.

\begin{figure}[H]
\begin{center}\includegraphics[scale=0.64]{rnn}\end{center}
\caption[rnn]{Štruktúra rekurentnej neurónovej siete~\cite{jaeger2002tutorial}}\label{fig:rnn}
\end{figure}

\subsection{Siete s dlhou krátkodobou pamäťou - LSTM}


LSTM predstavuje vylepšený model RNN. Vnútorná štruktúra ako doplnok ku externej rekurencii medzi jednotlivými neurónmi obsahuje aj \textit{internú rekurenciu}, zobrazenú v štruktúre LSTM neurónu na obr.~\ref{fig:lstm}. Medzi najdôležitejšie súčasti tohto modelu patria sigmoidné brány, ktoré rozhodujú o tom, ako sa signál bude širiť. LSTM tak prekonáva problém strácajúceho sa gradientu, ktorým trpí klasická RNN architektúra~\cite{hochreiter1997long}.
\newline
\textbf{Brána zabudnutia} ovplyvňuje, či nastáva vnútorná rekurencia neurónu. Stav tak môže ale nemusí byť faktorom ovplyvňujúcim nasledujúcu iteráciu výpočtu v sieti. Významné zlepšenie v LSTM sieťach prišlo s myšlienkou \textit{kontextom podmieneného zabúdania}. Takýto model sa ukazuje extrémne výhodným pri riešení problémov zahŕňajúcich \textit{časové pauzy}(lags)~\cite{gers2000learning}.
 Dôležitý prvok  na obr.~\ref{fig:lstm} predstavuje čierna kocka. Označuje pauzu o veľkosti jednej iterácie. Hodnota signálu tak ovplyvňuje nasledujúcu iteráciu, tj. vplýva na neskoršie udalosti.
\newline
\noindent


%ako do riti mam prelozit peepholes? 
\textbf{Nazeracie diery}(peepholes) predstavujú vylepšenie LSTM. Rieši problémy, ktoré vznikajú na základe faktu, že brána nedostáva priame informácie o stave jadra LSTM bloku(CEC). Táto situácia nastáva, keď je výstupná brána zatvorená. \textit{Nazeranie} predstavuje techniku váhovaného prepojenia CEC s bránami bloku daného jadra. Prepojenia sú štandardné s výnimkou časovej pauzy.

LSTM siete v praxi dokázali svoje schopnosti pri aplikácií na rôzne netriviálne dátové problémy. Pozornosť je kladená na frekventovanú časovú závislosť v dátach:
%deeplearning LSTM ma odkazy na konkretne projekty
\begin{my_itemize}
	\item{Rozoznávanie rukopisu} ~\cite{greff2015lstm}
	\item{Rozoznávanie reči} ~\cite{graves2013speech}
	\item{Označovanie obrázkov} ~\cite{kiros2014unifying}
\end{my_itemize}


\begin{figure}[H]
\begin{center}
\includegraphics[scale=0.50]{peepholes}\end{center}
\caption[peepholes]{Schéma nazerania v LSTM bloku~\cite{gers2003learning}}
\label{fig:lstm}
\end{figure}

\chapter{Dáta sprístupnené pre prácu}
\label{analyza_data}

Pre túto prácu boli sprístupnené dáta z používateľských prístupov na stránky e-shopu ponúkajúceho produkty a služby tretích strán za výhodnejších podmienok. Partneri(tretie strany) uverejňujú svoje ponuky na portáli, ktorý im ponúka zákaznícky atraktívne prostredie. Spoločnosti získavajú zákazníkov bez potreby nadmernej reklamnej kampane tým, že sa ich ponuka objaví na portáli s vysokou navštevovanosťou.

Pri všetkých používateľských prístupoch a aktivitách portál eviduje potenciálne dôležité dáta o týchto prístupoch pre možnosti dátovej analytiky a rozpoznávania trendov v správaní zákazníkov pre generovanie pridanej hodnoty pomocou prispôsobovania sa a individuálneho prístupu k zákazníkom. Takéto biznis stratégie vedú k maximalizovaniu zisku z predaja a zákazníckej spokojnosti pri práci s portálom. Dáta sú kategorizované a inak spracúvané do použiteľnej podoby. Výsledkom je dataset, ktorý slúži ako základ pre prácu odporúčacích systémov, ktoré táto práca skúma.

\section{Štruktúra dát}



\subsection{Sekvenčné dáta}

Po roztriedení používateľov podla unikátneho identifikátora dostávame sekvencie akcií používateľa na webe. \textbf{Sekvenčné dáta} predstavujú hlavný informačný vstup pre zvolenú metódu strojového učenia - rekurentné neurónové siete. Sú charakteristické tým, že zachytávajú chronologický sled udalostí v rámci definovanej domény. Sekvenčné dáta nemusia mať nutne diskrétny časový charakter - časové intervaly môžu byť rôzne medzi jednotlivými udalosťami a taktiež nie je podmienkou aby boli intervaly známe. Dôležité je zachovanie následnosti jednotlivých udalostí v čase.

\subsection{Obsah dát}
Medzi najdôležitejšie dostupné údaje z datasetu používateľských akcií patria:

\begin{my_itemize}
	\item {Cookie}
	\item {Používateľské ID}
	\item {Čas aktivity}
	\item {Prehliadaný obsah}
	\item {Typ akcie}
	\item {Kategória produktu}
	\item {Cenový rozsah ponuky}
	\item {Časový rozsah ponuky}
	\item {Dostupnosť ponuky}
	\item {Dodávateľ tretej strany}
\end{my_itemize}

\subsubsection*{Používateľské ID a cookie} 
Používateľ je v dátach identifikovateľný dvomi spôsobmi. Pokiaľ je používateľ registrovaný a prihlásený, jeho akcie sú spojené s jeho používateľským identifikátorom v databáze. Avšak títo používatelia predstavujú menšinu v dátach. V praxi väčšina e-shopov nevyžaduje registráciu pred nákupom a vzniká množstvo dát bez odkazu na registrovaného používateľa. Tieto akcie sú ukladané pod používateľskou cookie.

\textbf{Cookie} predstavuje unikátny identifikátor webového prehliadača voči internetu. Okrem identifikácie používateľa sa využíva na udržiavanie stavu v HTTP prostredí a po prijatí od webového servera sa ukladá na disk používateľa. Problematika cookie spočíva v tom, že jeden používateľ môže vystupovať pod rôznymi cookie identifikátormi. Takáto situácia môže byť spôsobená prístupom z rôznych zariadení alebo cieleným používateľským zásahom do cookie - zmazaním, po ktorom používateľ odosiela cez prehliadač odlišnú cookie identifikáciu. Akcie pri ktorých máme iba cookie však predstavujú väčšinu a ich vynechanie je stratou dát cenných pre algoritmy strojového učenia.


\subsubsection*{Používateľské sedenie}
\label{session}
Snahou dátového zberu je schopnosť rozpoznať charakteristiky zákazníka, ktorý pristupuje k produktom a službám. Jeden ucelený prístup k portálu sa nazýva \textbf{používateľské sedenie}(user session). Charakterizuje ho používateľ a jeho navigácia hierarchiou stránok a inými interakciami v rámci portálu.  Sedenie ohraničujúce jeden používateľský prístup je definované ako sekvencia akcií, trvajúca pokým nedojde ku neaktivite používateľa presahujúcej definovaný časový rámec. Štandardný rozsah pre toleranciu neaktivity je 30 minút.  ~\cite{zhou2010research}.


Dostupné dáta sa po technickej stránke delia na dve skupiny: 
\begin{my_itemize}
	\item {Metadáta} 
	\item {Behaviorálne dáta}
\end{my_itemize}

\subsection{Metadáta}

Metadáta sú databázová schéma obsahujúca štrukturované údaje o používateľoch, partneroch, ponukách, podponukách a iných dátových entitách, ktoré e-shop eviduje. 

\subsection{Behaviorálne dáta}
Behaviorálne dáta predstavujú záznamy používateľských akcií na portáli. Akcie pochádzajú z interakcie používateľa so stránkou. Každá akcia predstavuje záznam v databáze, s príslušným typom akcie, identifikátorom používateľskej cookie, príslušnou časovou pečiatkou zo vzniku akcie a ďalšími informáciami. Tieto informácie sa odkazujú na metadáta. \newline Medzi najpodstatnejšie typy akcií patria napríklad: 
\begin{my_itemize}
	\item {Zobrazenie produktov} 
	\item {Náhľad na ponuku}
	\item {Prezeranie hodnotení ponuky}
	\item {Pridanie do košíka}
	\item {Uskutočnenie nákupu}
\end{my_itemize}


\chapter{Výskum v danej oblasti}
\label{analyza_vyskum_danej_oblasti}

V tejto časti sa zaoberáme štúdiami skúmajúcimi aplikáciu data-miningu pri generovaní odporúčaní a aplikáciou neurónových sietí v dátovej analytike. Tieto štúdie vedú k optimalizácii poskytovania online produktov a služieb.


\subsection{Hlboká konvolučná neurónová sieť}
\label{metoda_konvolucna}


Pokročilé modely neurónových sietí ako hlboké konvolučné siete dosahujú veľmi dobré výsledky pri problémoch spracovania obrazu~\cite{szegedy2015going}.
Za účelom využitia týchto vlastností štúdia skladá z používateľskej aktivity obrazovú mapu - dvojrozmerné pole normalizovaných pixelov. Za účelom učenia má každý obraz dostupné označenie, ktoré hovorí či daný zákazník prešiel ku konkurencii alebo nie. Obr.~\ref{fig:pixelmap} zobrazuje ukážku aktivity zákazníka v dostupných službách za posledných $n$ dní~\cite{wangperawong2016churn}.

\begin{figure}[H]
\begin{center}
\includegraphics[scale=0.92]{pixelmap}\end{center}
\caption[pixelmap]{Aktivita zákazníka v mape pixelov. Hodnota pixelov sa zvyšuje od modrej k červenej
~\cite{wangperawong2016churn}.}
\label{fig:pixelmap}
\end{figure}

Experiment uvažuje 30-dňové okno predikcie, z ktorého sieť usudzuje aktivitu zákazníka. Okno sa nachádza 14 dní pred posledným registrovaný telefonátom. Pokiaľ sa posledný registrovaný telefonát nekonal v posledných 14 dňoch od aktuálneho dátumu, považujeme zákazníka za neaktívneho a neberieme ho do úvahy.
\newline
Po vytvorení obrazového datasetu z dostupných záznamov boli dáta podsunuté konvolučnej neurónovej sieti na obr.~\ref{fig:convolutional}.
Táto sieť má architektúru podobnú iným sieťam určeným pre spracovanie obrazu. Sieť analyzuje týždňové vzory v aktivite pomocou 7x1 filtra prvej konvolučnej vrstvy. Na konci siete je pomocou binárneho softmax klasifikátora vyhodnotený výsledok.
\newline

\begin{figure}[H]
\begin{center}
\includegraphics[scale=0.46]{convolutional}
\caption[convolutional]{Architektúra konvolučnej siete pre klasifikáciu zákazníkov z pixelovej mapy aktivity~\cite{wangperawong2016churn}.}
\label{fig:convolutional}
\end{center}
\end{figure}

Pomocou metódy \textit{oblasti pod krivkou}(AUC) bolo zistené, že konvolučná sieť dosahuje lepšie výsledku ako model CHAID rozhodovacieho stromu. AUC vyhodnocuje pravdivé aj nepravdivé pozitívne výsledky~\cite{hanley1982meaning}~\cite{bradley1997use}.

