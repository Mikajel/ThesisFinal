%%%%%%%%%%%%%%%%%%%%%%%%%%%%%%%%%%%%%%%%%%%%%%%%%%%%%%%%%%%%%%%%%%%%%%%%%%%%%%%%%%%%%%%%
%%
%% Slovak annotation
%%
%%%%%%%%%%%%%%%%%%%%%%%%%%%%%%%%%%%%%%%%%%%%%%%%%%%%%%%%%%%%%%%%%%%%%%%%%%%%%%%%%%%%%%%%
Odporúčacie systémy slúžia na generovanie pridanej hodnoty z dát, ktoré sa vytvárajú pri používateľských prístupoch k online službám a produktom. Takéto odporúčania dokážu zlepšiť online biznis aj zákaznícky zážitok z používania systému. Najnovšie prístupy skúmajú možnosti aplikácie metód strojového učenia pre zlepšenie týchto odporúčaní. 
V tejto práci sa zameriavam na možnosti využitia rekurentných neurónových sietí s dlhou krátkodobou pamäťou(LSTM) pri analýze dát z elektronických portálov sprostredkujúcich produkty a služby online. Takáto analýza poskytuje náhľad do používateľského správania a z toho vyplývajúce možnosti spätnej väzby voči návštevníkom online portálov. Manažment biznisu orientovaného na zákazníkov si čoraz viac uvedomuje cenu verného zákazníka na konkurenčnom trhu. Je preto nutné pri generovaní odporúčaní ponúknuť len to najrelevantnejšie a nezahlcovať zákazníka, čo ho odrádza od opätovnej návštevy portálu.

